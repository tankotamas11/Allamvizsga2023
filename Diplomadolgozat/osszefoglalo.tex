%----------------------------------------------------------------------------
\chapter*{Összefoglaló}\addcontentsline{toc}{chapter}{Összefoglaló}
%----------------------------------------------------------------------------

Dolgozatomban több gráfelméleti algoritmust megírtam és teszteltem. Valós adatokkal feltöltött adatbázist használtam, amit kis adatbányászattal és adatszűréssel alakítottam könnyen használható formába. Az adatokat egy egyszerű szöveges állományban tároltam, ami segítségemre lesz a gráfelméleti algoritmusok gyakorlásában az egyetemi tananyag részére.

Az adatokat szűrtem és rendeztem, majd külön figyelmet fordítottam az adatvizualizációra is. Úgy gondolom, hogy ha látom a gráfokat, könnyebben tudom megérteni azokat, és könnyebb meghatározni egy adott algoritmus eredményét vagy a valós értékét. A Python programozási nyelvet használtam az adatbányászathoz, szűrésekhez és a vizualizációhoz. Fájlkezeléseket, néhány reguláris kifejezést és a matplotlib osztályt használtam az ábrák megjelenítéséhez.

Ezután áttértem egy másik programozási nyelvre, mert úgy gondolom, hogy ha a gráfelméletet C/C++ nyelven tanuljuk, könnyebben megérthetjük és átláthatjuk hasonló nyelven megírt kódokat, mint egy idegen nyelven írt algoritmust. C++ nyelven megvalósítottam az algoritmusaimat és szoftvereimet, amik a szociális jelenségekhez kapcsolódnak. Például, a pletyka terjedését vizsgáló algoritmust vagy a csomósodási együttható kiszámítását, amivel meghatározhatjuk, hogy mennyi esély van a barátok összekapcsolódására. Az irányított és irányítatlan gráfokat is figyelembe vettem, és bemutattam a változásokat az idő múlásával.

Az egyetemi órákon szereztem meg a gráfokkal kapcsolatos információkat és ismereteket.\cite{Katai}

Ezen kutatást, amelynek során vizsgálatokat végeztem, szeretném a jövőben továbbfejleszteni és magasabb szintre emelni. Az egyik célom ennek a fejlesztésnek a mesterséges intelligencia bevonása és az üzenetek hangulatának elemzése. Ezen túlmenően hosszabb távú terveim között szerepel a mesterséges intelligencia alkalmazása után az akkori üzenetek összehasonlítása a jelenlegi cégekben vagy akár kisebb szervezetekben elküldött e-mailekkel.

Github link: \url{https://github.com/tankotamas11/Allamvizsga2023}