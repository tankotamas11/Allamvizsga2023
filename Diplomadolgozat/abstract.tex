\pagenumbering{gobble}

\selectlanguage{magyar}
\hungarianParagraph

%----------------------------------------------------------------------------
% Abstract in Hungarian
%----------------------------------------------------------------------------

\chapter*{Kivonat}
Dolgozatom témája a szociális jelenségek vizsgálata egy valós cég e-mail kommunikációs hálózatán, amely során különböző gráfokra alkalmazott algoritmusok futásidejét hasonlítom össze. Ezen műveletekkel összekapcsolom a szociális jelenségeket az informatikai algoritmikával és gráfelmélettel. Előnyös esetként nem egy kitalált (fiktív) adatbázis elemeivel dolgoztam, hanem az ENRON valós e-mail kommunikációs adatbázisán végeztem. A kutatás során adatbányászatot végeztem el, hogy kizárólag a vállalaton belüli kapcsolatokat vizsgáljam meg, ezzel csökkentve a gráfban használt csúcsok számát.

Az interdiszciplináris kutatásom során különböző gráfelméleti algoritmusokat használtam, például a mélységi bejárás és a Girvan-Newman módszer.

A valós adatbázis lehetőséget nyújt a hálózat dinamikájának vizsgálatára, ezáltal a kapcsolatok kialakulása is követhető, amit vizualizálással is megjelenítek.


\vfill
\selectlanguage{romanian}

%----------------------------------------------------------------------------
% Abstract in Romanian
%----------------------------------------------------------------------------
\chapter*{Rezumat}

Románra fordítva: Tema lucrării mele este investigarea fenomenelor sociale într-o rețea de comunicare prin email a unei companii reale, în cadrul căreia compar duratele de execuție ale algoritmilor aplicați asupra diferitelor grafuri. Prin aceste operațiuni, conectez fenomenele sociale cu algoritmica informatică și teoria grafurilor. Într-un scenariu favorabil, nu am lucrat cu elemente fictive, ci am utilizat baza de date reală a comunicărilor prin email ale companiei ENRON. În cadrul cercetării, am efectuat și minerit de date pentru a examina exclusiv relațiile interne din cadrul companiei, reducând astfel numărul de noduri utilizate în graf.

În timpul cercetării mele interdisciplinare, am utilizat diferite algoritme de teoria grafurilor, precum parcurgerea în adâncime și metoda Girvan-Newman.

Baza de date reală oferă posibilitatea de a investiga dinamica rețelei, permițându-ne să urmărim formarea conexiunilor, care pot fi vizualizate.

\vfill
\selectlanguage{english}
%\englishParagraph

%----------------------------------------------------------------------------
% Abstract in English
%----------------------------------------------------------------------------
\chapter*{Abstract}


The topic of my paper is the examination of social phenomena in a real company's email communication network, where I compare the execution times of algorithms applied to different graphs. Through these operations, I connect social phenomena with computer algorithms and graph theory. In a favorable scenario, I did not work with fictional elements, but used the real email communication database of the ENRON company. In the research, I also conducted data mining to exclusively examine internal relationships within the company, thereby reducing the number of nodes used in the graph.

During my interdisciplinary research, I utilized various graph-theoretical algorithms such as depth-first search and the Girvan-Newman method.


The real database provides the opportunity to examine the dynamics of the network, allowing us to track the formation of connections, which can be visualized as well.


\vfill
\dolgozatnyelve
\defaultParagraph